%%%%%%%%%%%%%%%%%%%%%%%%%%%%%%%%%%%%%%%%%%%%%%%%%%%%%%%%%%%%%%%%%%%%%%%%%%%%%%%%%%%
%% This project aims to create the UFC template for presentation.                %%
%% author: Maurício Moreira Neto - Doctoral student in Computer Science (MDCC)   %%
%% contacts:                                                                     %%
%%    e-mail: maumneto@ufc.br                                                    %%
%%    linktree: https://linktr.ee/maumneto                                       %%
%%%%%%%%%%%%%%%%%%%%%%%%%%%%%%%%%%%%%%%%%%%%%%%%%%%%%%%%%%%%%%%%%%%%%%%%%%%%%%%%%%%
\documentclass{libs/ufc_format}
% Inserting the preamble file with the packages
\input{libs/preamble.tex}
% Inserting the references file
\bibliography{references.bib}

% Title
\title[short title of the pre]{\huge\textbf{Title of the presentation}}
% Subtitle
\subtitle{Subtitle of the Presentation}
% Author of the presentation
\author{Name of Author}
% Institute's Name
\institute[UFC]{
    % email for contact
    \normalsize{\email{name@mail.sustech.edu.cn}}
    \newline
    % Department Name
    \department{Name of the Department}
    \newline
    % university name
    \SUSTech
}
% date of the presentation
\date{\today}


%%%%%%%%%%%%%%%%%%%%%%%%%%%%%%%%%%%%%%%%%%%%%%%%%%%%%%%%%%%%%%%%%%%%%%%%%%%%%%%%%%
%% Start Document of the Presentation                                           %%               
%%%%%%%%%%%%%%%%%%%%%%%%%%%%%%%%%%%%%%%%%%%%%%%%%%%%%%%%%%%%%%%%%%%%%%%%%%%%%%%%%%
\begin{document}
% insert the code style
\input{libs/code_style}

%% ---------------------------------------------------------------------------
% First frame (with tile, subtitle, ...)
\begin{frame}{}
    \maketitle
\end{frame}

%% ---------------------------------------------------------------------------
% Second frame
\begin{frame}{Table of Content}
    \begin{multicols}{2}
        \tableofcontents
    \end{multicols}
\end{frame}

%% ---------------------------------------------------------------------------
% This presentation is separated by sections and subsections
% \section{Section I}
% \begin{frame}{Explicações}
%     % itemize
%     Este é um template que pode ser utilizado para:
%     \begin{itemize}
%         \item Apresentação de Trabalhos Acadêmicos
%         \item Apresentação de Disciplinas
%         \item Apresentações de Teses e Dissertações
%     \end{itemize}

%     \vspace{0.4cm} % vertical space
    
%     % enumeration
%     Para utilizar este template corretamente é importante que:
%     \begin{enumerate}
%         \item Tenha conhecimento mínimo sobre LaTeX
%         \item Ler os comentários no template (explicações)
%         \item Ler o README.md (documentação)
%     \end{enumerate}

%     \vspace{0.2cm}

%     \example{Este é um texto de exemplo!} \emph{Texto de Ênfase!}
% \end{frame}

%% ---------------------------------------------------------------------------
\subsection{SubSection I}
\begin{frame}{Blocks}
    % Blocks styles
    \begin{block}{Block I}
        Text in the block.
    \end{block}

    \begin{alertblock}{Block II}
        Text in the alert block.
    \end{alertblock}

    \begin{exampleblock}{Block III}
        Text in the example block.
    \end{exampleblock}   
\end{frame}

%% ---------------------------------------------------------------------------
\subsection{SubSection II}
\begin{frame}{Different boxs}
    \successbox{Text in success box}

    \pause

    \alertbox{Text in alert box}

    \pause

    \simplebox{Text in simple box}
\end{frame}

%% ---------------------------------------------------------------------------
\subsection{SubSection III}
\begin{frame}{Insert Algorithm}
    \begin{algorithm}[H]
        \SetAlgoLined
        \LinesNumbered
        \SetKwInOut{Input}{input}
        \SetKwInOut{Output}{output}
        \Input{x: float, y: float}
        \Output{r: float}
        \While{True}{
          r = x + y\;
          \eIf{r >= 30}{
           ``O valor de $r$ é maior ou iqual a 10.''\;
           break\;
           }{
           ``O valor de $r$ = '', r\;
          }
         } 
         \caption{Algorithm Example}
    \end{algorithm}
\end{frame}

%% ---------------------------------------------------------------------------

\begin{frame}{Insert code}
    \lstset{language=Python}
    \lstinputlisting[language=Python]{code/main.py}
\end{frame}

%% ---------------------------------------------------------------------------
\begin{frame}{Insert cod}
    \lstinputlisting[language=C]{code/source.c}
\end{frame}

%% ---------------------------------------------------------------------------
\begin{frame}{Insert cod}
    \lstinputlisting[language=Java]{code/helloworld.java}
\end{frame}

%% ---------------------------------------------------------------------------
\begin{frame}{Insert cod}
    \lstinputlisting[language=HTML]{code/index.html}
\end{frame}

%% ---------------------------------------------------------------------------
% This frame show an example to insert multicolumns

%% ---------------------------------------------------------------------------
% This frame show an example to insert figures
\section{Images}
\begin{frame}{Section III - Figures}
    \begin{figure}
        \centering
        \caption{SUSTech LOGO.}
        \includegraphics[scale=0.3]{libs/LOGO.png}
        \source{SUSTech LOGO}
        \label{fig:ufc_emblem}
    \end{figure}
\end{frame}

%% ---------------------------------------------------------------------------
% Reference frames
\begin{frame}[allowframebreaks]
    \frametitle{References}
    \printbibliography
\end{frame}

%% ---------------------------------------------------------------------------
% Final frame
\begin{frame}{}
    \centering
    \huge{\textbf{\example{Thanks!}}}
    
    \vspace{1cm}
    
    \Large{\textbf{Contact:}}
    \newline
    \vspace*{0.5cm}
    \large{\email{Cnatact info}}
\end{frame}

\end{document}